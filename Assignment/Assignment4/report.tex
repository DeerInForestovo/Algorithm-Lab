\documentclass{article}

\usepackage{ctex}
\usepackage{graphicx}
\usepackage{float}
\usepackage{datetime}
\usepackage{hyperref}
\usepackage{amssymb}
\usepackage{amsthm}
\newtheorem{theorem}{Theorem}
\newtheorem{corollary}{Colollary}
\newtheorem{lemma}{Lemma}
\newtheorem{definition}{Definition}
\renewcommand\proofname{Proof}
\renewcommand\figurename{Figure}
\usepackage{listings}
\usepackage{xcolor}
\usepackage[linesnumbered, ruled]{algorithm2e}
\usepackage[legalpaper, margin=1.4in]{geometry}

\title{CS216 Assignment4 \\ {\begin{large} Description of the subtree disassembly trick proposed by Tarjan \end{large}}}  %title

\author{匡亮(12111012)}

\date{May 1, 2023}  % date

\begin{document}

\maketitle

\renewcommand\abstractname{Abstract}
\begin{abstract}

This is my report of CS216 Assignment4, including a description of the subtree disassembly trick proposed by Tarjan:

\begin{itemize}
    \item[1.] Its core idea and critical procedures.
    \item[2.] Running time and space complexity analysis.
    \item[3.] Its pseudocode.
    \item[4.] How it speeds up SPFA.
\end{itemize}

\end{abstract}

\newpage % contents page

\renewcommand\contentsname{Contents}
\tableofcontents

\newpage  % 1st page here

\section{Core idea and critical procedures of Tarjan's trick}

For convenience, in the whole report we assume the terminus vertex $t$ is reachable from every vertex in $G$, and our algorithm will immediately terminate if we find a negative ring (though some vertices cannot reach it, which means they still have a valid non-infinite minimum distance to the terminus vertex). Under these conditions, Tarjan's trick can significantly speed up SPFA without change the algorithm a lot.

Here I don't repeat what we have learned in our courses (like how SPFA works) and begin from SPFA.

In SPFA, we maintain an array $first$ for each vertex, and check whether there is a ring after $n$ iterations to determine whether there is a negative ring. However, if we have found a ring during the iterations, we can immediately terminate the algorithm and report there is a negative ring. Therefore, the basic idea of this trick is try to detect whether there is a ring when modifying $first$ with an acceptable extra cost.

Obviously, before we find a ring, $(v, first[v]),\forall v\not=t$ is a tree with root $t$. When we set $first[v]\leftarrow w$, we can do DFS in $v$'s subtree and check whether $w$ is in $v$'s subtree, if so, we can terminate our algorithm. However, this DFS will cost an extra $O(n)$ time, so we should expect it to do more for us. We can further observe that before the modification, the minimum distance of all vertices in $v$'s subtree (call them $x$) is evaluated based on the minimum distance of $v$, which has been just modified. Therefore, the minimum distance of all of them are out of date. We can set them \textbf{dormant}, then they will not be used to update others' minimum distance before their own minimum distance are updated. At the same time we can remove them from the tree (set $first[x]\leftarrow null$), so that they will not be set dormant by other ancestors again. They will add themselves back to the tree automatically when their minimum distance are updated (because they will update $first[x]$ at that time).

\section{Running time and space complexity analysis}

Based on SPFA, we just store one more tree with $O(1)$ space per vertex, so the space complexity is still $O(n)$.

Add/Remove a vertex to/from the tree can be easily done in $O(1)$ time with linked-list. Adding only happens when $first[v]$ are updated for some $v$, and removing happens no more than adding because vertices will not be removed twice, so the time complexity is still $O(nm)$.

\newpage

\section{Pseudocode}

\begin{algorithm}
    \KwData{G, t}
    $n=$ number of nodes in $G$\;
    Array $M[V]$\;
    Initialize $M[t]=0$ and $M[v]=\infty$ for all other $v\in V$\;
    Initialize $first[t]=t$ and $first[v]=null$ for all other $v\in V$\;
    Initialize a tree with only root vertex $t$\;
    $foundNegativeRing=$ False\;
    \For {$i=1,...,n-1$} {
        \For {$w\in V$ where $w$ is not dormant} {
            \If {$M[w]$ has been updated in the previous iteration} {
                \For {all edges $(v,w)$} {
                    $M[v]=\min(M[v],c_{vw}+W[w])$\;
                    \If {this change the value of $M[v]$} {
                        $first[v]=w$\;
                        \For {all vertex $x$ in $v$'s subtree} {
                            \If {$x==w$} {
                                $foundNegativeRing=$ True\;
                                End the algorithm\;
                            }
                            Set $x$ dormant\;
                            Remove $x$ from the tree\;
                        }
                        Set $v$ not dormant\;
                        Add $v$ to the tree as $w$'s child\;
                    }
                }
            }
        }
        \If {no value changed in this iteration} {
            End the algorithm\;
        }
        \tcc { If $foundNegativeRing$ is true, then users should ignore the returned $M,first$. }
    }
    \KwResult{$foundNegativeRing,M,first$}
    \caption{SPFA Improved by Tarjan's Trick}
\end{algorithm}

\section{How it speeds up SPFA}

If there is a negative ring, the algorithm will terminate in advance, so it is speeded up without a doubt. However, if there is no negative rings, this trick can still speed SPFA up.



\newpage  % reference page

\renewcommand\refname{References}  % change '&' in the link to '\&'
\begin{thebibliography}{99}
    \bibitem[Jon Kleinberg / Éva Tardos(2005)]{textbook} Algorithm Design (P.304 - 307)

    \bibitem[Stefan Lewandowski(2010)]{article} Shortest Paths and Negative Cycle Detection in Graphs with Negative Weights - I: The Bellman-Ford-Moore Algorithm Revisited
    https://d-nb.info/1014960916/34

\end{thebibliography}

\end{document}
